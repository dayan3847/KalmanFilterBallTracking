\begin{frame}
    \Huge
    \begin{center}
        \textbf{Conditional Probability}
    \end{center}
\end{frame}

\subsection{Definition}\label{subsec:definition2}
\begin{frame}
    \frametitle{Definition}
    \begin{block}{}
        In probability theory, conditional probability is a measure of the probability of an event occurring, given that another event (by assumption, presumption, assertion or evidence) has already occurred.
    \end{block}
\end{frame}
\begin{frame}
    \frametitle{Definition}
    \begin{block}{}
        The conditional probability of an event $A$ given that another event $B$ has occurred is the probability of $A$ given $B$:
        \begin{equation}
            P[A|B] = \frac{P[A \cap B]}{P[B]}\label{eq:equation21}
        \end{equation}
    \end{block}
\end{frame}
