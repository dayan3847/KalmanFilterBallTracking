%! Author = Dayan Bravo Fraga <dayan3847@gmail.com>

\documentclass[24pt]{article}

% Preamble
\usepackage{graphicx}
%\usepackage[utf8]{inputenc}
%\usepackage{geometry}
%\usepackage{amsfonts}
%\usepackage{amssymb}
%\usepackage{caption}
%\usepackage{subcaption}
%%\usepackage{cite}
%\usepackage{hyperref}
\usepackage{authblk}
\usepackage{amsmath}
\usepackage{cancel}

% Front

% Title
\title{
    \includegraphics[scale=.1]{./img/logo_uady}\\
    \normalfont \large \textsc{Bayes Estimation} \\ [0.3cm]
    \Large \bfseries{Practice 2} \\
    \HRule \\[0.1cm]
    \small \textit{Analytical demonstration of $\mu$ recursive and iterative formula.}
}
% Author
\author{
    \textsc{Ing.}~Dayan~\textsc{Bravo~Fraga}~\textsuperscript{1}\\
    \emph{Teacher:}~\textsc{Dr.}~Arturo~\textsc{Espinosa~Romero}~\textsuperscript{2}\\
}
% Date
\date{September 2023}
% Document
\begin{document}
    % Title
    \maketitle
    \begin{center}
        \textbf{Universidad Autónoma de Yucatán} \textit{Facultad de Matemáticas}\\
    \end{center}


    \section{Demonstration}\label{sec:demonstration}
    %es:
    %Demostración analítica de quivalencia de las formulas de mu iterativa y recursiva.
    %en:
    Analytical demonstration of $\mu$ iterative and recursive formula equivalence.

    \subsection{Definitions}\label{subsec:definitions}
    %es:
    %Supongamos que tienes una secuencia de números {x1, x2, x3, ..., xn} que quieres promediar, donde n es el número de elementos en la secuencia.
    %Formula iterativa:
    %en:
    Suppose you have a sequence of numbers ${x_1, x_2, x_3, \ldots, x_n}$ that you want to average, where $n$ is the number of elements in the sequence.

    \subsection{Iterative formula}\label{subsec:iterative-formula}
    %es:
    %La fórmula iterativa para calcular el promedio de una secuencia de números es:
    %en:
    The iterative formula for calculating the average of a sequence of numbers is:

    \begin{equation}
        \mu_{n} = \frac{1}{n} \sum_{i=1}^{n} x_i\label{eq:equation2}
    \end{equation}
    or
    \begin{equation}
        \mu_{n} = \frac{x_1 + x_2 + x_3 + \dots + x_n}{n}\label{eq:equation1}
    \end{equation}

    \subsection{Recursive formula}\label{subsec:recursive-formula}
    %es:
    %La fórmula recursiva para calcular el promedio de una secuencia de números es:
    %en:
    The recursive formula for calculating the average of a sequence of numbers is:
    \begin{equation}
        \mu_{n} =
        \begin{cases}
            x_n & \text{if } n = 1 \\
            \frac{(n-1)\mu_{n-1} + x_n}{n} & \text{if } n > 1
        \end{cases}
        \label{eq:equation3}
    \end{equation}

    \subsection{Induction}\label{subsec:induction}
    %es:
    % Demostrar que estas dos fórmulas son equivalentes utilizando inducción matemática.
    %en:
    Prove that these two formulas are equivalent using mathematical induction.

    \subsection{Base step $n = 1$}\label{subsec:base-step}
    %es:
    %Demostrar que la fórmula recursiva es válida para el caso base, es decir, cuando n = 1.
    %en:
    Prove that the recursive formula is valid for the base case, i.e.~when $n = 1$.

    \begin{align}
        \intertext{Iterative Formula and Recursive Formula when $n = 1$}
        \frac{x_1}{n} &= x_n \\
        \intertext{Substitute $n = 1$ in both formulas}
        \frac{x_1}{1} &= x_1 \\
        \intertext{Verify}
        x_1 &= x_1
    \end{align}

    \subsection{Inductive step $n > 1$}\label{subsec:inductive-step}
    %es:
    %Demostrar que la fórmula recursiva es válida para el paso inductivo, es decir, cuando n > 1.
    %en:
    Prove that the recursive formula is valid for the inductive step, i.e.~when $n > 1$.

    %es:
    %Supongamos que las fórmulas iterativa y recursiva son equivalentes para n = k., es decir, que:
    %en:
    Suppose the iterative and recursive formulas are equivalent for $n = k$, i.e.~that:

    \begin{align}
        \mu_{k} &= \frac{1}{k} \sum_{i=1}^{k} x_i \\
        \mu_{k} &= \frac{x_1 + x_2 + x_3 + \dots + x_k}{k} \\
    \end{align}

    %es:
    %Demostrar que las fórmulas iterativa y recursiva son equivalentes para n = k + 1, es decir, que:
    %en:
    Prove that the iterative and recursive formulas are equivalent for $n = k + 1$, i.e.~that:

    \begin{align}
        \mu_{k+1} &= \frac{1}{k+1} \sum_{i=1}^{k+1} x_i \\
        \mu_{k+1} &= \frac{x_1 + x_2 + x_3 + \dots + x_k + x_{k+1}}{k+1} \\
    \end{align}

    \begin{align}
        \intertext{Iterative Formula and Recursive Formula when $n = k + 1$}
        \frac{x_1 + x_2 + x_3 + \dots + x_k + x_{k+1}}{k+1} &= \frac{k \cdot \mu_{k} + x_{k+1}}{k+1} \\
        %es:
        %\intertext{Como suponemos que las fórmulas iterativa y recursiva son equivalentes para n = k, entonces podemos sustituir la fórmula iterativa de &\mu_{k} en la fórmula recursiva para obtener:}
        %en:
        \intertext{Since we assume that the iterative and recursive formulas are equivalent for $n = k$, then we can substitute the iterative formula of $\mu_{k}$ in the recursive formula to obtain:}
        \frac{x_1 + x_2 + x_3 + \dots + x_k + x_{k+1}}{k+1} &= \frac{k \cdot \frac{x_1 + x_2 + x_3 + \dots + x_k}{k} + x_{k+1}}{k+1} \\
        \intertext{Simplify}
        \frac{x_1 + x_2 + x_3 + \dots + x_k + x_{k+1}}{k+1} &= \frac{\cancel{k} \cdot \frac{x_1 + x_2 + x_3 + \dots + x_k}{\cancel{k}} + x_{k+1}}{k+1} \\
        \intertext{Verify}
        \frac{x_1 + x_2 + x_3 + \dots + x_k + x_{k+1}}{k+1} &= \frac{x_1 + x_2 + x_3 + \dots + x_k + x_{k+1}}{k+1}
    \end{align}

    \subsection{Conclusion}\label{subsec:conclusion}
    %es:
    %Se ha demostrado utilizando el método de inducción matemática.
    %Hemos demostrado que si las fórmulas son equivalentes para $n = k$, entonces también lo son para $n = k + 1$.
    %Dado que ya hemos demostrado que son equivalentes para $n = 1$ (paso base),
    %podemos concluir que las fórmulas iterativa y recursiva para calcular $\mu$ son equivalentes para cualquier valor de $n$,
    %por lo tanto, son igualmente válidas.
    %en:
    It has been demonstrated using the method of mathematical induction.
    We have shown that if the formulas are equivalent for $n = k$, then they are also equivalent for $n = k + 1$.
    Since we have already shown that they are equivalent for $n = 1$ (base step),
    we can conclude that the iterative and recursive formulas for calculating $\mu$ are equivalent for any value of $n$,
    therefore, they are equally valid.

\end{document}